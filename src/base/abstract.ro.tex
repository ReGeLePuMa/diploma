% This file contains the abstract of the thesis

Containerele sunt o tehnologie populară pentru rularea aplicațiilor într-un mediu consistent și izolat.
Cea mai populară platformă de containerizare este Docker, care permite crearea imaginilor de container prin definirea unui set de instrucțiuni într-un fișier text simplu numit \textit{Dockerfile}.
Aceste imagini, însă, nu conțin doar codul aplicației, ci și software suplimentar care ajută dezvoltatorii să gestioneze și să interacționeze cu containerul.
Această teză explorează procesul de extragere a set-ului minim de fișiere necesare pentru rularea aplicației containerizate, aplicând tehnici de analiză statică, dinamică și prin forță brută, cu scopul de a crea un Dockerfile capabil să genereze o astfel de imagine minimă.
În plus, investigăm avantajele utilizării unor astfel de imagini minimale în mediile de producție, precum și examinăm scenarii specifice de utilizare a acestor Dockerfile-uri minimale, în special în ceea ce privește proiectul Unikraft.