% This file contains the abstract of the thesis

Containers are a popular technology for deploying applications in a consistent and isolated environment.
The most popular containerization platform is Docker, which allows for the creation of container images by
defining a set of instructions in a simple text file called a \textit{Dockerfile}.
These images, however, do not contain only application code, but additional software that helps developers
manage and interact with the container.
This thesis explores the process of extracting the the minimal set of files needed to run the containerized application by
applying static, dynamic and brute force analysis techniques with the goal of creating a Dockerfile capable of creating such an image.
Additionally, we also investigate the advantages of using such minimal images in production environments as well as examine specific use
case scenarios for the creation of minimal Dockerfiles, especially in regards to the Unikraft project.
