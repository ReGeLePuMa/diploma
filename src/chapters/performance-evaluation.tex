\chapter{Performance Evaluation}
\label{chapter:performance-evaluation}

In order to evaluate the minimization process, I tried minimizing a wide range of applications, from simple \textit{Hello World} programs to complex 
software like \textit{Wordpress}, \textit{Chromium} and \textit{MariaDB}, summing a sample size of 
38. For each application, I calculated the size reduction of the minimal Docker image obtained by the minimization process compared to the 
original Docker image as well as the stage it reached in the minimization process (\textit{Static Analysis}, \textit{Dynamic Analysis} or \textit{Binary Search\footnote{I chose to limit the steps of the binary search to 3, due to time constraints and fear of hardware failure }}).
The results are as follows:

\fig[scale=0.5]{src/img/pie_chart.pdf}{img:dockerminimizer-performance}{Distribution of the results by stage of minimization}[2pt]

The chart above shows that the program was able to correctly minimize applications in all three stages and that the majority of the applications were minimized in the 
\textit{Static Analysis} and \textit{Dynamic Analysis} stages, with only the more complex applications requiring the \textit{Binary Search} stage.

In terms of the size reduction per stage, the results are the following:

\fig[scale=0.5]{src/img/reduction_chart.pdf}{img:dockerminimizer-size-reduction}{Size reduction per stage of minimization}[2pt]

As displayed in the chart above, the applications that benefited the most from the minimzation process were those that were able to be minimized in the \textit{Static Analysis} stage, with an average size reduction of 96\%. The applications that required the \textit{Dynamic Analysis} stage had an average size reduction of 80\%, while those that required the \textit{Binary Search} stage had an average size reduction of 17\% ( with 3 iterations ).
The average size reduction across all applications was 70\%, which is a significant improvement over the original Docker images.
For a full breakdown of the size reduction per application, see the chart below: 

\fig[width=0.8\textwidth, height=0.3\textheight]{src/img/all_stage_reductions.pdf}{img:dockerminimizer-size-reduction-per-app}{Size reduction per application}[2pt]

The main issues that I identified with the application were related to the \textit{Dynamic Analysis} stage, where the application was not able to correctly identify all the files that were not used by the application. This was due to the fact that some applications require a more complex analysis of their dependencies and usage patterns, 
which \textit{ptrace} that \textit{strace} uses is not able to provide.
Another issue was the relatively high execution time of the \textit{Binary Search} stage, which can take a long time to complete for larger applications, owing to the archiving of many files and building of the images per each iteration,
which is the reason why I limited the number of iterations to 3.
Overall, the application was able to successfully minimize a wide range of applications, with an average size reduction of 70\%.