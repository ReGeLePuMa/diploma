\chapter{Background}
\label{chapter:background}

\section{A brief history of containers}
\label{sec:history}
The idea of containerization is not a new one. The concept has it's roots
since the late 70s, with \textit{chroot}, a Unix command that allows a process to change its root directory,
effectively isolating it from the rest of the system \cite{history-of-containers}, creating
so called \textit{chroot jails}. Over the decades, this concept grew and evolved, with the introduction
of \textit{FreeBSD jails} and \textit{Solaris Zones} around the turn of the millennium adding support for multiple 
isolated environments within the same OS instance \cite{history-of-containers}. 

The next major milestone happened in 2008 with the introduction of \textit{LXC} (Linux Containers),
adding kernel-level support for containers, by leveraging two Linux kernel components: \textit{cgroups}, which
provides ways to group processes in order to better manage resources and \textit{namespaces}, which provide isolation.\cite{history-of-containers}

In 2013, the launch of \textit{Docker} radically changed the landscape of containerization by providing a developer-friendly way to 
create, manage and deploy containers. Docker introduced the concept of container images, files which store the data needed for the container to run, 
\href{https://hub.docker.com/}{\textit{Docker Hub}}, a public repository for sharing container images, and a powerful command-line interface for managing containers \cite{history-of-containers}.
In the years that followed, multiple platforms sprung up such as \textit{Apptainer} and \textit{Podman}, which are both open-source alternatives to Docker, but none of them managed to gain the same level of popularity.
As such, Docker became synonymous with containerization, holding an overwhelming market share of \textit{87.85}\%\cite{docker-market-share} thus cementing
themselves as the de facto standard in the industry.

\section{The anatomy of a container}
\label{sec:anatomy}


\fig[scale=0.5]{src/img/containers-vs-vms.png}{img:containers-vs-vms}{Containers vs VMs}

As opposed to virtual machines, which run a full operating system on top of a hypervisor, containers share the host operating system and run as isolated processes
making them much more lightweight and efficient by only having to package the application and its dependencies into a so-called container image.
Currently, there are over 10000 container images available on \href{https://hub.docker.com/}{Docker Hub}, ranging from simple \href{https://hub.docker.com/_/hello-world}{\textit{hello-world}} apps, to 
\href{https://hub.docker.com/search?categories=Databases+\%26+Storage}{\textit{databases}} and even full-blown Linux distributions like \href{https://hub.docker.com/_/ubuntu}{\textit{Ubuntu}}.

Now, how do we create our own container image? By creating a \textit{Dockerfile}, a simple text file that contains all the necessary steps to build the image.\\\\

\lstset{language=Dockerfile,caption=Sample Dockerfile,label=lst:ex-dockerfile}
\begin{lstlisting}
FROM python:3.10.0-slim
COPY requirements.txt .
RUN pip install -r requirements.txt
COPY . .
CMD ["python", "adapter.py"]    
\end{lstlisting}

The format of a Dockerfile is quite simple, with each line representing a command that will be executed in order, the specifications for what each command does can be found in the official documentation\footnote[1]{\href{https://docs.docker.com/reference/dockerfile/}{https://docs.docker.com/reference/dockerfile/}}
As each command is executed, a new layer is applied on top of the previous one, the container image being represented as a stack of these layers. A layer represents a change to the base image layer such as adding or removing files, installing packages or changing environment variables.

\fig[scale=0.25]{src/img/container-as-layers.png}{img:container-as-layers}{Container layers}

All Dockerfiles require a base image, which is specified by the keyword \textit{FROM}.
The base image can be represented by either an existing image, or if you want to create your own, you can use \textit{\textbf{FROM} scratch} to start from an empty filesystem.
Additionally, Docker supports building the container image in multiple stages, by using the \textit{FROM} command multiple times in the same Dockerfile.

\lstset{language=Dockerfile,caption=Sample Multi-stage Dockerfile,label=lst:ex-multi-stage-dockerifle}
\begin{lstlisting}
    FROM alpine:latest AS builder
    RUN apk --no-cache add build-base
    
    FROM builder AS build1
    COPY source1.cpp source.cpp
    RUN g++ -o /binary source.cpp
    
    FROM builder AS build2
    COPY source2.cpp source.cpp
    RUN g++ -o /binary source.cpp   
\end{lstlisting}

This coupled with the \textit{scratch} base image serves as the mechanism that allows for the creation of minimal container images, by defining first a 
builder image and then copying the required files to the empty image.

\section{Related work}
\label{sec:related-work}

Before proceeding further, it is important to acknowledge the work done by 
my peers in the field of container image minimization.

The earliest mention of building minimal container images is in a blog post dated 4th of July 2014 by
Adriaan de Jonge on the site Xebia \cite{xebia-post} and is corroborated by a 
2015-02-03\footnote{not sure if it's the 2nd of March or 3rd of February as I could not find this presentation and the only mention of this is in his GitHub repository \cite{minimal-containers-101}} München talk by Brian Harrington \cite{minimal-containers-101}
which describe using \textit{tar} to create the \textit{scratch} image and then copying the required files to it, or by using 
tools like \textit{buildroot}\footnote{\href{https://buildroot.org/}{Buildroot}} or \textit{debootstrap}\footnote{\href{https://wiki.debian.org/Debootstrap}{debootstrap}} to create a minimal Linux distribution and then copying the required files to it.

In 2016, the first Alpine Linux image was published, which had a compressed size of 1.86MB\footnote{\href{https://hub.docker.com/layers/library/alpine/2.6/images/sha256-e9cec9aec697d8b9d450edd32860ecd363f2f3174c8338beb5f809422d182c63}{Oldest Alpine Linux image on Docker Hub}} as it
was built using \textit{musl} and \textit{busybox}, two lightweight alternatives to the standard C library and coreutils respectively.
Around the same time, the \textit{Distroless} philosophy i.e images containing only the application and its dependencies, was introduced by Google, with the 
first of these images being build using the open source tool \textit{Bazel} for languages such as Java, Go and Python.
Recently, Canonical, the company behind Ubuntu, has embraced distroless with their \textit{Chiseled}\footnote{
    \href{https://ubuntu.com/containers/chiseled}{Chiseled}
} ubuntu images.

However, none of these tackle the fundamental issue of using minimal container images,
being that the developer has to manually determine the dependencies of the application,
which can be a arduous task, especially for large applications. The only project that I could find 
that is tangentially related to this is \textit{dockershrink} by developer Raghav Dua \cite{dockershrink}.
This tool utilizes Artificial Intelligence in order reduce the size of the container image by generating a new
Dockerfile updated to use slimmer base images and removing unnecessary files. \cite{dockershrink}
However, this tool is still in beta and it only works for NodeJS applications, as well as having to
provide it with a OpenAI API key in order to access the full functionality of the tool.
Additionally, it does not utilize the \textit{scratch} image, which means that the final 
container image generated by the shrunk Dockerfile is not the most minimal possible.

 