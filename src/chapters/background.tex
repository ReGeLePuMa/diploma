\chapter{Background}
\label{chapter:background}

\section{A brief history of containers}
\label{sec:history}
The idea of containerization is not a new one. The concept has it's roots
since the late 70s, with \textit{chroot}, a Unix command that allows a process to change its root directory,
effectively isolating it from the rest of the system \cite{history-of-containers}, creating
so called \textit{chroot jails}. Over the decades, this concept grew and evolved, with the Introduction
of \textit{FreeBSD jails} and \textit{Solaris Zones} around the turn of the millennium adding support for multiple 
isolated environments within the same OS instance \cite{history-of-containers}. 

The next major milestone happened in 2008 with the introduction of \textit{LXC} (Linux Containers),
adding kernel-level support for containers, by leveraging two Linux kernel components: \textit{cgroups}, which
provides ways to group processes in order to better manage resources and \textit{namespaces}, which provide isolation.\cite{history-of-containers}

In 2013, the launch of \textit{Docker} radically changed the landscape of containerization by providing a developer-friendly way to 
create, manage and deploy containers. Docker introduced the concept of container images, files which store the data needed for the container to run, 
\textit{Docker Hub}, a public repository for sharing container images, and a powerful command-line interface for managing containers \cite{history-of-containers}.
In the years that followed, Docker became synonymous with containerization, holding an overwhelming market share of \textit{87.85}\%\cite{docker-market-share} thus cementing
themselves as the de facto standard for containerization.


