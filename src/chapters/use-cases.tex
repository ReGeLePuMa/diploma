\chapter{Use Cases}
\label{chapter:use-cases}

A real world example for the need to generate these minimal Dockerfiles and
the reason that spawned this project is \textit{Unikraft}\footnote[1]{\url{https://unikraft.org/}}.

Unikraft was envisioned as a faster and more secure alternative to running applications in containers or virtual machines, by
leveraging the power of ultra-lightweight virtual machines knows as unikernels \cite{unikraft}.

\fig[scale=1]{src/img/vm-container-unikernel.pdf}{img:vm-container-unikernel}{High-level comparison of the software components of traditional VMs (a), containers (b), containers in VMs (c) with unikernels (d). \cite{unikraft}}[2pt]

Unikernels are specialized operating systems that are built as single-address space binary objects, meaning that they don't 
have user space-kernel space separation. They combine both the complete execution isolation and hardware access typical of VMs with the size of containers,
which makes them ideal for cloud deployments  \cite{unikraft}.

\begin{table}[ht]
\centering
\small
\begin{tabular}{|l|p{3cm}|p{3cm}|p{3cm}|}
\hline
\textbf{Property} & \textbf{Virtual Machines} & \textbf{Containers} & \textbf{Unikernels} \\
\hline
Time performance & Slowest of the 3 & Fast & Fast \\
\hline
Memory footprint & Heavy & Depends on the number of features & Light \\
\hline
Security & Very secure & Least secure of the 3 & Very secure \\
\hline
Features & Everything you could think of & Depends on the needs & Only the absolute necessary \\
\hline
\end{tabular}
\caption{Comparison between Virtual Machines, Containers, and Unikernels \cite{unikraft}}
\label{tab:vm-container-unikernel-comparison}
\end{table}


Unikraft provides the tools needed to build, run and manage unikernels, in the form of a command-line program called \textit{kraft}\footnote{\url{https://unikraft.org/docs/cli}}.
It features an interface similar to that of Docker's, with it being able to run pre-build unikernels or create new ones from a file called \textit{Kraftfile}\footnote{also \textit{kraft.yaml} or \textit{kraft.yml} for legacy support}.

\lstset{language=yaml,caption=Sample Kraftfile,label=lst:ex-kraftfile}
\begin{lstlisting}
    spec: v0.6
    runtime: base:latest
    rootfs: ./Dockerfile
    cmd: ["/helloworld"]  
\end{lstlisting}

Kraftfiles are written in YAML\footnote{\url{https://yaml.org/}}, which follows an attribute-value structure. For my thesis, the only relevant attribute is \textit{rootfs}, which
describes the source from which the unikernel's root filesystem is built. Although optional \cite{unikraft-filesystem}, since most applications do require one, it is usually specified.
As unikernels operate in resource-constrained specialized environments where efficiency is crucial, so do their filesystems need to strike the balance between
providing the necessary storage requirements and also adhering to their lightweight philosophy.

There are many ways to define a root filesystem, one of which being highlighted in the Kraftfile example above, where a Dockerfile is used to build a container whose filesystem is later extracted to be used
by the resulting unikernel. Thus, the Dockerfile should be as minimal as possible. So far, this process was done manually\footnote{
    \url{https://unikraft.org/docs/contributing/adding-to-the-app-catalog}
}, which explains the rather low number of applications that have been ported to Unikraft\footnote{
    \url{https://github.com/unikraft/catalog}
}, which need constant maintaining as different versions of the same app may require different Dockerfiles. 