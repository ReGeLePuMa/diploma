\chapter{Architecture Overview}
\label{chapter:architecture}

\section{Static Analysis}
\label{sec:static-analysis}

Static analysis is the process of analyzing a program's code without executing it. In our context, we will be using it
to identify the libraries that the application is dynamically linked using the \textit{ldd} command.
\begin{lstlisting}[language=bash, caption=ldd command, label=lst:ex-ldd]
    ~ > ldd /usr/bin/man
        linux-vdso.so.1 (0x00007ffe831f5000)
        libmandb-2.9.1.so => /usr/lib/man-db/libmandb-2.9.1.so (0x00007f068b572000)
        libman-2.9.1.so => /usr/lib/man-db/libman-2.9.1.so (0x00007f068b52f000)
        libz.so.1 => /lib/x86_64-linux-gnu/libz.so.1 (0x00007f068b502000)
        libpipeline.so.1 => /lib/x86_64-linux-gnu/libpipeline.so.1 (0x00007f068b4f1000)
        libc.so.6 => /lib/x86_64-linux-gnu/libc.so.6 (0x00007f068b2ff000)
        libgdbm.so.6 => /lib/x86_64-linux-gnu/libgdbm.so.6 (0x00007f068b2ef000)
        libseccomp.so.2 => /lib/x86_64-linux-gnu/libseccomp.so.2 (0x00007f068b2cb000)
        /lib64/ld-linux-x86-64.so.2 (0x00007f068b59b000)
\end{lstlisting}

The output of the command is a list of shared libraries that the application depends on, along with their paths.
This approach works well for simple applications but it quickly becomes insufficient for most production applications.

\section{Dynamic Analysis}
\label{sec:dynamic-analysis}

Dynamic analysis is the process of analyzing a program's behavior during its execution. Our use case is to identify the other files 
that the application accesses during its execution, as well any spawned processes and their dependencies.
This is done using the \textit{strace} command, which uses the \textit{ptrace} system call to trace the system calls made by a process.
\begin{lstlisting}[language=bash, caption=strace command, label=lst:ex-strace]
    ~ > strace -fe execve,openat echo "Hello, World\!"
    execve("/usr/bin/echo", ["echo", "Hello, World!"], 0x7fff43b04ce8 /* 36 vars */) = 0
    openat(AT_FDCWD, "/etc/ld.so.cache", O_RDONLY|O_CLOEXEC) = 3
    openat(AT_FDCWD, "/lib/x86_64-linux-gnu/libc.so.6", O_RDONLY|O_CLOEXEC) = 3
    Hello, World!
    +++ exited with 0 +++
\end{lstlisting}

For our scenario, we are interested in the \textit{openat}, which means that the application is trying to open a file, and \textit{execve}, which means that the application is trying to execute another program.
The \textit{-f} flag is used to allow strace to also trace the child processes spawned by the application, thus 
ensuring that we do not omit any dependencies.

\section{Brute Force}
\label{sec:brute-force}

The brute force approach is a last resort method that is used should both the static and dynamic analysis fail in generating the minimal Dockerfile.
Using the Docker SDK, we can extract the filesystem from the container and use it for this step. Given that in the containerized environment, the application is
running normally and in a \textit{FROM scratch} environment it is not (which does not contain any files), therefore there exists a point where removing a file from the filesystem causes
the application to fail. Since removing files one at a time and building a new container each time is not feasible, we have to attempt a more efficient approach, that being a \textit{binary search}.

\fig[scale=1]{src/img/brute-force.png}{img:brute-force}{Brute force approach to finding the minimal Dockerfile}

We start by splitting in half the filesystem into two buckets - \textit{used} and \textit{unused}. The decision of which files land in what bucket will be 
left to a coin toss in order to ensure fairness. We then build the Dockerfile with the files in the \textit{used} bucket and run the application. If it fails, then we split the \textit{unused} bucket in half and repeat the process.
If the app succeeds, we repeat the cycle by splitting the shrunken filesystem in half again and follow the same steps. This will repeat at max \textit{MAX_LIMIT} times, a predefined limit we impose on the number of iterations.
This sifting process is a greedy one and thus will not always generate the minimal Dockerfile, but it will offer a good approximation of it, especially if the limit is set to a high value.
