\chapter{Motivation and Objectives}
\label{chap:motivation-and-objectives}

The main advantages of minimal containers are their enhanced security and
small image size.
Their improved security comes from the inherent properties of being minimal, meaning:
\begin{itemize}
    \item \textbf{minimized attack surface} - by having only the required dependencies for the application to run,
the only attack vector a hacker has is the application itself and not other components
    \item \textbf{clear dependency tree} - with only the required dependencies, it is easier to identify and
audit them in case of a vulnerability being discovered,
\cite{minimal-containers}
\end{itemize}
Additionally, their small size means that they are faster to deploy and use fewer system resources like \textit{CPU} and \textit{memory},
which is especially important in a cloud environment where the cost is directly proportional to the resources used and shaving a
couple of seconds off the deployment time can lead to hours, even days given the size of the cluster. \cite{container-deployment}.

By creating a tool that can automatically detect an application's dependencies and create the Dockerfile which produces the minimal container for that app,
we can save developers the time and effort of having to do it themselves, which can be tiresome and frustrating process.